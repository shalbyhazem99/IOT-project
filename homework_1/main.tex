\documentclass{article}

\usepackage[italian]{babel}     %testi autogenerati italiano
\usepackage{graphicx}           %per importare immagini
\usepackage{geometry}           %per gestire margini e spostamenti
\usepackage[raggedright]{titlesec}
\geometry {
    top=20mm,
    bmargin=20mm,
}
\usepackage{array}              %per colonne di width fissata
\usepackage{subcaption}         %tabelle divise
\usepackage{hyperref}           %links
\hypersetup{
    colorlinks=true,
    linkcolor=black,
    urlcolor=blue
}
%\usepackage[bottom]{footmisc}   %footnotes fissate a piè pagina
\usepackage{booktabs}           %per tabitem in tabular
\newcommand{\tabitem}{~~\llap{\textbullet}~~}
\renewcommand*{\thefootnote}{[\arabic{footnote}]}

\begin{document}

\setlength\parindent{0pt} %noindent automatico
\setlength\parskip{1em}

\begin{titlepage}
	\centering
	\hrule
	
	\vspace{6,5cm}
	{\Huge \textbf{Home Challenge \#1\\
		2020/21}\\}
		
		\vspace{0,5cm}
		\large {Prof. Cesana Matteo}
		
		\vspace{2,5cm}
		{
			\large
			\begin{tabular}{c c}
				Shalby Hazem Hesham Yousef & (Personal Code: 10596243) \\
			\end{tabular}
			
		}
		\vspace{4cm}
		
		\normalsize{6 April 2021}
		\vspace{0,2cm}
		
		\centering\hspace{0,2cm}\includegraphics[scale=0.6]{./logo.png}
		\vspace{0,5cm}
		\hrule
		
		\end{titlepage}
		
		\pagebreak
		
		
		
		\pagebreak
		
		\section{Questions} %1
		
		\subsection{What’s the difference between the message with MID: 3978 and the one with MID: 22636?} %1.1
		The main difference between the two messages is that one is reliable (\textbf{MID: 3978}) and one is unreliable (\textbf{MID: 22636}). The reliability of the first message is ensured by CONFIRMABLE (\textbf{CON}) Message (no. 6701) which must be acknowledged through \textbf{ACK} Message (no. 6702). The second message is unreliable in fact the exchange is done via a NON-CONFIRMABLE (\textbf{NON}) message (no. 6943). The filter used to get all the information is:
		\begin{center}
			\texttt{coap.mid == 3978 || coap.mid == 22636}
		\end{center}
		
		\subsection{Does the client receive the response of message No. 6949?} %1.2
		\textbf{YES}. Using \texttt{frame.number == 6949} filter we get a CONFIRMABLE message with \texttt{MID: 28357} so we should have an ACK message with the same ID as response.\\
		Using the \texttt{coap.mid == 28357 \&\& coap.type == 2} filter we get a message as result so the client received the response (\texttt{No. 6953}).
		
		\subsection{How many replies of type confirmable and result code “Content” are received by the server “localhost”?} %1.3
		There are \textbf{8} messages that respect the request. The result is obtained by using the following filter:
		\begin{center}
			\texttt{ip.src == 127.0.0.1 \&\& coap.type == 2 \&\& coap.code == 69}
		\end{center}
		
		\subsection{How many messages containing the topic “factory/department*/+” are published by a client with user name: “Jane”? \footnote{* replaces the dep. number, e.g. factory/department1/+, factory/department2/+ and so on.}} %1.4
		\textbf{Zero}.
		Using \texttt{mqtt.username == "jane"} filter I got 4 CONNECT messages related to four TCP stream with the following index:
		\begin{itemize}
			\item \texttt{112}
			\item \texttt{121}
			\item \texttt{230}
			\item \texttt{354}
		\end{itemize}
		Using the previous information and the filter:
		\begin{center}
			\texttt{(tcp.stream == 112 || tcp.stream == 121 || tcp.stream == 230 || tcp.stream == 354 ) \&\& mqtt.msgtype == 3 \&\& mqtt.topic contains "factory/department"}
		\end{center}
		I got no message that respect the request.
		
		\subsection{How many clients connected to the broker “hivemq” have specified a will message?} %1.5
		\textbf{10}. Using \texttt{dns.resp.name contains "hivemq"} filter we got \texttt{2} addresses as response:
		\begin{itemize}
			\item \texttt{18.185.199.22};
			\item \texttt{3.120.68.56}.
		\end{itemize}
		And using \texttt{(ip.addr == 18.185.199.22 || ip.addr == 3.120.68.56) \&\& mqtt.willmsg} we got 10 messages as result.
		
		\subsection{How many publishes with QoS 1 don’t receive the ACK?} %1.6
		\textbf{50}. Using \texttt{mqtt.msgtype == 3 \&\& mqtt.qos == 1} filter we got \texttt{124} messages published with \texttt{QoS 1}.
		The reception of messages with \texttt{QoS 1} is confirmed by \texttt{PUBACK} message, but using \texttt{mqtt.msgtype == 4} we got only \texttt{74} messages, so there are \texttt{124-74= 50} messages that didn’t receive any ACK.
		
		\subsection{How many last will messages with QoS set to 0 are actually delivered?} %1.7
		\textbf{ONE}. Using \texttt{mqtt.msgtype==1 \&\& mqtt.conflag.qos ==0 \&\& mqtt.willmsg} I got 14 connect messages that set the Will Message and only one is actually delivered.
		
		\subsection{Are all the messages with QoS \textgreater{ 0} published by the client 4m3DWYzWr40pce6OaBQAfk correctly delivered to the subscribers?} %1.8
		\textbf{ONE}.
		Using \texttt{mqtt.clientid == 4m3DWYzWr40pce6OaBQAfk} filter I got one massage (\texttt{no.964}) and this indicates only one connection by the client analyzed. Following the TCP connection and filtering using the filter:
		\begin{center}
			\texttt{tcp.stream eq 67 \&\& mqtt.qos>0 \&\& mqtt.msgtype==3}
		\end{center}
		I got 2 messages: \texttt{No.1008} and \texttt{No.2423}.\\
		The first has \texttt{QoS == 1} so it’s correctly delivered if and only if a \texttt{PUBACK} is received, but using the filter:
		\begin{center}
			\texttt{tcp.stream eq 67 \&\&  mqtt.msgtype == 4}
		\end{center}
		
		I got nothing.\\
		The second has \texttt{QoS == 2} so it’s correctly received if and only if a \texttt{PUBREC} is received, and this happens, in fact using the filter: \begin{center}
		\texttt{tcp.stream eq 67 \&\&  mqtt.msgtype == 5}
		\end{center} 
		I got one message (No.2425)
		
		
		\subsection{What is the average message length of a connect msg using mqttv5 protocol? Why messages have different size?} %1.9
		\textbf{The average length is 91 or 32}. The message have different length because the length of the options are different (username, password, etc...). For filtering the CONNECT messages that relay on MQTTv5, we use the following filter: 
		\begin{center}
			\texttt{mqtt.ver == 5 \&\& mqtt.msgtype == 1}
		\end{center}
		I proposed two answers because the question is unclear because it doesn't specify if the average length needed is the one of the entire frame (\texttt{91}) or only of the MQTT(\texttt{32}).
		
		\subsection{Why there aren’t any REQ/RESP pings in the pcap?} %1.10
		The request/response mechanism is used to prevent clients with very low messages rates from being disconnected. In the pcap considered there aren't any REQ/RESP because the keepalive timer never expires.
		\section{Filters Reference}
		\begin{itemize}
			\item \texttt{coap.type == 2} $\rightarrow$ COAP message related to an ACK
			\item \texttt{coap.code == 69} $\rightarrow$ COAP message with result code equals to "Content"
			\item \texttt{mqtt.msgtype == 1} $\rightarrow$ MQTT connect message
			\item \texttt{mqtt.msgtype == 3} $\rightarrow$ MQTT publish message
			\item \texttt{mqtt.msgtype == 4} $\rightarrow$ MQTT publish acknowledge (PUBACK) message
			\item \texttt{mqtt.msgtype == 5} $\rightarrow$ MQTT publish receive (PUBREC) message
			          
			          
		\end{itemize}
		
		\pagebreak
		\clearpage
\end{document}