\documentclass{article}

\usepackage[italian]{babel}     %testi autogenerati italiano
\usepackage{graphicx}           %per importare immagini
\usepackage{geometry}           %per gestire margini e spostamenti
\usepackage[raggedright]{titlesec}
\geometry {
    top=20mm,
    bmargin=20mm,
}
\usepackage{array}              %per colonne di width fissata
\usepackage{subcaption}         %tabelle divise
\usepackage{hyperref}           %links
\hypersetup{
    colorlinks=true,
    linkcolor=black,
    urlcolor=blue
}
%\usepackage[bottom]{footmisc}   %footnotes fissate a piè pagina
\usepackage{booktabs}           %per tabitem in tabular
\newcommand{\tabitem}{~~\llap{\textbullet}~~}
\renewcommand*{\thefootnote}{[\arabic{footnote}]}

\begin{document}

\setlength\parindent{0pt} %noindent automatico
\setlength\parskip{1em}

\begin{titlepage}
	\centering
	\hrule
	
	\vspace{6,5cm}
	{\Huge \textbf{Home Challenge \#2\\
		2020/21}\\}
		
		\vspace{0,5cm}
		\large {Prof. Cesana Matteo}
		
		\vspace{2,5cm}
		{
			\large
			\begin{tabular}{c c}
				Shalby Hazem Hesham Yousef & (Personal Code: 10596243) \\
			\end{tabular}
			
		}
		\vspace{4cm}
		
		\normalsize{19 April 2021}
		\vspace{0,2cm}
		
		\centering\hspace{0,2cm}\includegraphics[scale=0.6]{./logo.png}
		\vspace{0,5cm}
		\hrule
		
		\end{titlepage}
		
		\pagebreak
		
		\pagebreak
		
		\section{Thing Speak} %1
		The ThingSpeak channel used for the challenges is: \href{https://thingspeak.com/channels/1359553}{ThingSpeak Channel}
	    
	    \section{NODE-RED} %2
		The Node-red application could be divided into three parts:
			\begin{itemize}
			\item \textbf{PARSING DATA}: the data are read from the CSV file and using a function block they are then formatted.\footnote{The provided CSV Node of NODE-RED is not used because the provided file is not well formatted.}
			\item \textbf{ELABORATING DATA}: here only the publish messages are taken into consideration and the correct field is attached to every data.
			\item \textbf{PUBLISHING RESULTS}: the MQTT message is created and then published\footnote{The messages rate is 1 message/minute to avoid any problem with ThingSpeak}.
            \end{itemize}
		
		\section{Results} %3
		Running the Node-red application on the given CSV file, I got 49 results which are divided as follows:
		
		\begin{itemize}
		    \item \texttt{20} for topic \texttt{factory/department1/section1/plc}
		    \item \texttt{0} for topic \texttt{factory/department3/section3/plc}
		    \item \texttt{13} for topic \texttt{factory/department1/section1/hydraulic\_valve}
		    \item \texttt{16} for topic \texttt{ factory/department3/section3/hydraulic\_valve}

		\end{itemize}
		The results grouped by field are the following: 
		\begin{itemize}
		    \item \textbf{Field 1}: \texttt{4,403,21,66,66,66,764,32,32,36,36,5,66,1747,4,31,764,14,2010,1380}
		    \item \textbf{Field 2}: \texttt{2,1344,14,638,14,1344,60,11,559,30,42,20,3162,14,195,2,14,14,14,\\3162,3162,1,1,39,14,1344,1344,3162,39}
		\end{itemize}
		
		\section{CSV Assumptions} %4
		The only assumption made is that the hex value at the end of every line appear in the same order of the publish message.\\\\
		\pagebreak
		\pagebreak
		\clearpage
\end{document}